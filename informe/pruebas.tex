\section{Pruebas y Validación}

Esta sección describe las pruebas realizadas para verificar el correcto funcionamiento de la topología, el controlador SDN y las reglas del firewall.

\subsection{Configuración del Entorno de Pruebas}

Todas las pruebas se realizaron siguiendo el siguiente procedimiento:

\begin{enumerate}
    \item \textbf{Iniciar el controlador} (Terminal 1):
    \begin{verbatim}
    ./run_controller.sh
    \end{verbatim}
    
    \item \textbf{Iniciar la topología} (Terminal 2):
    \begin{verbatim}
    ./run_topology.sh 2    # Para N=2 switches
    \end{verbatim}
    
    \item \textbf{Esperar a que los switches se conecten} y las reglas se instalen.
\end{enumerate}

\textbf{Importante}: El controlador debe iniciarse \textit{antes} que la topología para que los switches puedan conectarse al momento de su creación.

\subsection{Verificación Básica: Conectividad}

\subsubsection{Prueba 1: PingAll}

\textbf{Objetivo}: Verificar que todos los hosts pueden comunicarse entre sí (excepto donde el firewall lo impide).

\textbf{Comando}:
\begin{verbatim}
mininet> pingall
\end{verbatim}

\textbf{Resultado Esperado}:
\begin{itemize}
    \item $h_1 \leftrightarrow h_2$: Conectividad exitosa
    \item $h_1 \leftrightarrow h_3$: Conectividad exitosa
    \item $h_1 \leftrightarrow h_4$: Conectividad exitosa
    \item $h_2 \leftrightarrow h_3$: Bloqueado (Regla 3)
    \item $h_2 \leftrightarrow h_4$: Conectividad exitosa
    \item $h_3 \leftrightarrow h_4$: Conectividad exitosa
\end{itemize}

\textbf{Logs del Controlador}:
\begin{verbatim}
[Insertar captura de logs mostrando instalación de reglas]
\end{verbatim}

\textbf{Captura de Mininet}:
\begin{figure}[H]
    \centering
    \includegraphics[width=0.8\textwidth]{resultado_pingall.png}
    \caption{Resultado de pingall}
    \label{fig:resultado_pingall}
\end{figure}

\textbf{Capturas de Wireshark}:
Captura de interfaz de s1 que conecta con h2:
\begin{figure}[H]
    \centering
    \includegraphics[width=0.8\textwidth]{resultado_pingall_h2.png}
    \caption{Resultado de pingall para h2}
    \label{fig:resultado_pingall_h2}
\end{figure}
Se puede observar que se envia un mensaje de ping a h3 pero no se recibe respuesta.
Tampoco se recibe un ping de h3.\\

Captura de interfaz de s2 que conecta con h3:
\begin{figure}[H]
    \centering
    \includegraphics[width=0.8\textwidth]{resultado_pingall_h3.png}
    \caption{Resultado de pingall para h3}
    \label{fig:resultado_pingall_h3}
\end{figure}
En este caso no se recibe respuesta del ping a h2 ni se recibe un ping de este ultimo.\\

Captura de interfaz de s1 que conecta con s2:
\begin{figure}[H]
    \centering
    \includegraphics[width=0.8\textwidth]{resultado_pingall_switches.png}
    \caption{Resultado de pingall en los switches}
    \label{fig:resultado_pingall_switches}
\end{figure}
Se observa que no se envian los pings de h2 a h3 y viceversa ya que estos son descartados.

\subsection{Pruebas de Firewall}

\subsubsection{Prueba 2: Bloqueo de Puerto 80 (HTTP) - Regla 1}

\textbf{Objetivo}: Verificar que todo el tráfico al puerto 80 es bloqueado.

\textbf{Configuración}:
\begin{verbatim}
# Terminal h1 - Iniciar servidor HTTP en puerto 80
mininet> h1 python3 -m http.server 80 &

# Terminal h4 - Intentar acceder desde otro host
mininet> h4 curl -m 5 http://10.0.0.1:80
\end{verbatim}

\textbf{Resultado Esperado}: Timeout (conexión bloqueada por firewall)

\textbf{Resultado Obtenido}:
\begin{figure}[H]
    \centering
    \includegraphics[width=0.8\textwidth]{resultado_curl_h4-h1.png}
    \caption{Resultado de curl de h4 a h1}
    \label{fig:resultado_curl_h4-h1}
\end{figure}

\textbf{Verificación con iperf}:
\begin{verbatim}
# Servidor en h1
mininet> h1 iperf -s -p 80 &

# Cliente en h4
mininet> h4 iperf -c 10.0.0.1 -p 80 -t 5
\end{verbatim}

\textbf{Resultado}: No se establece conexión.
\begin{figure}[H]
    \centering
    \includegraphics[width=0.8\textwidth]{resultado_iperf_h4-h1.png}
    \caption{Resultado de iperf de h4 a h1 en puerto 80}
    \label{fig:resultado_iperf_h4-h1}
\end{figure}

\textbf{Prueba de Control (puerto no bloqueado)}:
\begin{verbatim}
mininet> h1 python3 -m http.server 8000 &
mininet> h4 curl http://10.0.0.1:8000
\end{verbatim}

\textbf{Resultado}: Conexión exitosa (el puerto 8000 no está bloqueado).
\begin{figure}[H]
    \centering
    \includegraphics[width=0.8\textwidth]{resultado_curl_8000_h4-h1.png}
    \caption{Resultado de curl de h4 a h1 en puerto 8000}
    \label{fig:resultado_curl_8000_h4-h1}
\end{figure}

\subsubsection{Prueba 3: Bloqueo UDP Específico - Regla 2}

\textbf{Objetivo}: Verificar que el tráfico UDP desde $h_1$ al puerto 5001 es bloqueado.

\textbf{Configuración}:
\begin{verbatim}
# Servidor UDP en h4
mininet> h4 iperf -s -u -p 5001 &

# Cliente desde h1 (bloqueado)
mininet> h1 iperf -c 10.0.0.4 -u -p 5001 -t 5

# Cliente desde h2 (permitido)
mininet> h2 iperf -c 10.0.0.4 -u -p 5001 -t 5
\end{verbatim}

\textbf{Resultados Esperados}:
\begin{itemize}
    \item Desde $h_1$: 0\% de paquetes recibidos (bloqueado)
    \item Desde $h_2$: 100\% de paquetes recibidos (permitido)
\end{itemize}

\textbf{Resultados Obtenidos}:
\begin{verbatim}
[Insertar salida de iperf mostrando los porcentajes]
\end{verbatim}

\subsubsection{Prueba 4: Bloqueo Bidireccional - Regla 3}

\textbf{Objetivo}: Verificar que $h_2$ y $h_3$ no pueden comunicarse en ninguna dirección.

\textbf{Test 1: ICMP (ping)}
\begin{verbatim}
mininet> h2 ping -c 4 h3
mininet> h3 ping -c 4 h2
\end{verbatim}

\textbf{Resultado Esperado}: 100\% packet loss en ambas direcciones.

\textbf{Test 2: TCP (iperf)}
\begin{verbatim}
mininet> h3 iperf -s &
mininet> h2 iperf -c 10.0.0.3 -t 5
\end{verbatim}

\textbf{Resultado Esperado}: No se establece conexión TCP.

\textbf{Resultado Obtenido}:
\begin{verbatim}
[Insertar salida mostrando que no hay conectividad]
\end{verbatim}

\textbf{Prueba de Control}:
\begin{verbatim}
mininet> h2 ping -c 4 h1    # Debería funcionar
mininet> h3 ping -c 4 h4    # Debería funcionar
\end{verbatim}

\textbf{Resultado}: Conectividad exitosa (solo está bloqueada la comunicación entre h2 y h3).

\subsection{Pruebas con Diferentes Topologías}

\subsubsection{Topología con N=1}

\textbf{Configuración}:
\begin{verbatim}
./run_topology.sh 1
\end{verbatim}

\textbf{Características}: Todos los hosts conectados al mismo switch (topología estrella).

\textbf{Resultado}: Las reglas del firewall se aplican correctamente independientemente de la topología.

\subsubsection{Topología con N=3}

\textbf{Configuración}:
\begin{verbatim}
./run_topology.sh 3
\end{verbatim}

\textbf{Características}: Cadena de 3 switches, aumenta la distancia entre hosts extremos.

\textbf{Resultado}: 
\begin{itemize}
    \item L2 learning funciona correctamente en topología extendida
    \item Firewall mantiene las reglas en todos los switches
    \item Latencia aumenta proporcionalmente con el número de saltos
\end{itemize}

\subsection{Análisis de Tráfico con Wireshark}

% TODO: Agregar capturas de Wireshark mostrando:
% - Paquetes ICMP Echo Request/Reply
% - Mensajes OpenFlow (OFPT_FLOW_MOD, OFPT_PACKET_IN)
% - Tráfico TCP/UDP bloqueado vs permitido

\begin{figure}[H]
    \centering
    % \includegraphics[width=0.9\textwidth]{wireshark_packet_in.png}
    \textit{[Insertar captura de Wireshark mostrando PacketIn]}
    \caption{Mensaje PacketIn de OpenFlow al controlador}
    \label{fig:wireshark_packetin}
\end{figure}

\begin{figure}[H]
    \centering
    % \includegraphics[width=0.9\textwidth]{wireshark_flow_mod.png}
    \textit{[Insertar captura de Wireshark mostrando FlowMod]}
    \caption{Mensaje FlowMod instalando regla de firewall}
    \label{fig:wireshark_flowmod}
\end{figure}
