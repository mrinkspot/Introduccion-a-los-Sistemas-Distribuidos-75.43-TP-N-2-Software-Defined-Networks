\section{Preguntas a responder}

\subsection{¿Cuál es la diferencia entre un switch y un router? ¿Qué tienen en común?}

Los switches y los routers son dispositivos esenciales para que una red funcione.
Ambos reciben paquetes, toman decisiones de reenvío y permiten que distintos equipos se comuniquen.
\\

Sin embargo, presentan diferencias claras:

\begin{enumerate}
    \item \textbf{Capa en la que operan:}
    Un switch trabaja en la capa de enlace de datos, utiliza direcciones MAC y una tabla de conmutación para decidir por qué puerto enviar cada frame.
    Un router opera en la capa de red, utiliza direcciones IP y una tabla de enrutamiento para determinar la mejor ruta hacia otras redes.

    \item \textbf{Tipo de comunicación que habilitan:}
    El switch conecta dispositivos dentro de la misma red local (LAN).
    El router conecta diferentes redes entre sí (LAN–LAN, LAN–Internet), permitiendo alcanzar destinos externos.

    \item \textbf{Reenvío de datos:}
    El switch aprende qué dirección MAC está asociada a cada puerto y envía el frame únicamente al puerto correspondiente.
    El router requiere configuración IP y emplea algoritmos de enrutamiento para decidir la ruta óptima. Al procesar información de la capa de red, el reenvío suele ser más costoso computacionalmente.

    \item \textbf{Funciones de seguridad y control:}
    El switch ofrece controles básicos a nivel de puerto.
    El router puede aplicar reglas más avanzadas (como NAT o firewall), gracias a que analiza información de las capas de red y transporte.
\end{enumerate}

En cuanto a sus similitudes:

\begin{enumerate}
    \item Ambos son dispositivos de conmutación de paquetes (store-and-forward).
    \item Ambos toman decisiones de reenvío utilizando una tabla interna (MAC table / routing table).
    \item Forman parte crítica de la infraestructura de red.
    \item En un entorno SDN, ambos pueden administrarse desde un controlador central que define cómo deben manejar el tráfico.
\end{enumerate}
